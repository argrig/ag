Пусть $V, U$ - векторные пространства,
$W = T_1 \wedge \ldots \wedge T_k \in \wedge^k(V \otimes U)$ задает $k$-мерное
подпространство в $V \otimes U$, где $T_i \in V \otimes \U$.
Пусть $V^*$ и $U^*$ - двойственные к $V$ и $U$ пространства.
Теорема. $W$ является тензорным произведение подпространств размерностей $n$
и $m$ тогда и только тогда, когда
1) размерность $\dim \{ f \in V^* \mid$ \text{свертка (по первым индексам) тензора $f \otimes W$ равна $0 \} = \dim (V) - n$ и
2) размерность $\dim \{ g \in U^* \mid \text{свертка (по первым индексам) тензора
$g \otimes W$ равна $0$ \} = \dim (U) - m$.
 Условия 1) и 2) несложно записать в виде алгебраических уравнений для $W$.
